\documentclass[12pt,a4paper,twoside]{article}

\usepackage{amsmath,amssymb}
\usepackage[utf8]{inputenc}                                      
\usepackage[OT4]{fontenc}      
%\usepackage[T1]{fontenc}                            
\usepackage[polish]{babel}                           
\selectlanguage{polish}
\usepackage{indentfirst} 
\usepackage[dvips]{graphicx}
\usepackage{tabularx}
\usepackage{color}
\usepackage{hyperref} 
\usepackage{fancyhdr}
\usepackage{listings}
\usepackage{booktabs}
\usepackage{ifpdf}
\usepackage{mathtext} % polskie znaki w trybie matematycznym
\usepackage{lmodern}
\usepackage{filecontents}
\usepackage{ifthen}
\usepackage{tikz}
\usetikzlibrary{arrows}
\newcounter{nextYear}
\setcounter{nextYear}{\the\year}
\stepcounter{nextYear}
\newcommand{\tl}[1]{\textbf{#1}} 
\pagestyle{fancy}
\renewcommand{\sectionmark}[1]{\markright{\thesection\ #1}}
\fancyhf{} % usuwanie bieżących ustawień
\fancyhead[LE,RO]{\small\bfseries\thepage}
\fancyhead[LO]{\small\bfseries\rightmark}
\fancyhead[RE]{\small\bfseries\leftmark}
\renewcommand{\headrulewidth}{0.5pt}
\renewcommand{\footrulewidth}{0pt}
\addtolength{\headheight}{0.5pt} % pionowy odstęp na kreskę
\fancypagestyle{plain}{%
\fancyhead{} % usuń p. górne na stronach pozbawionych numeracji
\renewcommand{\headrulewidth}{0pt} % pozioma kreska
}
\lstset{%
language=C++,%
commentstyle=\textit,%
identifierstyle=\textsf,%
keywordstyle=\sffamily\bfseries, %
tabsize=3,%
frame=lines,%
numbers=left,%
numberstyle=\tiny,%
numbersep=5pt,%
breaklines=true,%
morekeywords={pWezel,Wezel,string,ref,params_result},%
escapeinside={(*@}{@*)},%
\let\oldmarginpar\marginpar
\renewcommand\marginpar[1]{%
  {\linespread{0.85}\normalfont\scriptsize%
\oldmarginpar[\hspace{1cm}\begin{minipage}{3cm}\raggedleft\scriptsize\color{black}\textsf{#1}\end{minipage}]%    left pages
{\hspace{0cm}\begin{minipage}{3cm}\raggedright\scriptsize\color{black}\textsf{#1}\end{minipage}}% right pages
}%
}
\newcounter{rok}
\newcommand{\rokakademicki}{%
   \setcounter{rok}{\number\year}%
   \ifthenelse{\number\month<10}%
   {\addtocounter{rok}{-1}}% rok akademicki zaczal sie w pazdzierniku poprzedniego roku
   {}%                       rok akademicki zaczyna sie w pazdzierniku tego roku
   \arabic{rok}/\addtocounter{rok}{1}\arabic{rok}
}
\usepackage{color}
\definecolor{brickred}      {cmyk}{0   , 0.89, 0.94, 0.28}

\makeatletter \newcommand \kslistofremarks{\section*{Uwagi} \@starttoc{rks}}
\newcommand\l@uwagas[2]
{\par\noindent \textbf{#2:} %\parbox{10cm}
   {#1}\par} \makeatother
\newcommand{\ksremark}[1]{%
   {{\color{brickred}{[#1]}}}%
   \addcontentsline{rks}{uwagas}{\protect{#1}}%
}
\newcommand{\comma}{\ksremark{przecinek}}
\newcommand{\nocomma}{\ksremark{bez przecinka}}
\newcommand{\styl}{\ksremark{styl}}
\newcommand{\ortografia}{\ksremark{ortografia}}
\newcommand{\fleksja}{\ksremark{fleksja}}
\newcommand{\pauza}{\ksremark{pauza `--', nie dywiz `-'}}
\newcommand{\kolokwializm}{\ksremark{kolokwializm}}
\newcommand{\cytowanie}{\ksremark{cytowanie}}
\fancyhead[RE]{\small\bfseries Wojciech Janota} % autor sprawozdania
\begin{document}
\frenchspacing
\thispagestyle{empty}
\begin{center}
{\Large\sf Politechnika Śląska   % Alma Mater

Wydział Informatyki, Elektroniki i Informatyki

}

\vfill

 

\vfill\vfill

{\Huge\sffamily\bfseries Programowanie Komputerów 2\par}  

\vfill\vfill

{\LARGE\sf Steganografia w pliku BMP}   


\vfill \vfill\vfill\vfill

%%%%%%%%%%%%%%%%%%%%%%%%%%%%





\begin{tabular}{ll}
	\toprule
	autor                       & Wojciech Janota     \\
	prowadzący                  & dr inż. Mirosław Skrzewski  \\
	rok akademicki              & 2019/2020         \\
	kierunek                    & informatyka            \\
	rodzaj studiów              & SSI                    \\
	semestr                     & 2                      \\
	termin laboratorium         & środa, 15:30 -- 17:00 \\
	sekcja                      & 51                     \\
	termin oddania sprawozdania & 2020-07-23             \\
	\bottomrule
	                            &
\end{tabular}

\end{center}
%%% koniec strony  tytulowej

%%%%%%%%%%%%%%%%%%%%%%%%%%%%%%%%%%%%%%%%%%%%%%%%%%%%%%%%%%%%%%%%%%%%%%%%%
\cleardoublepage
%%%%%%%%%%%%%%%%%%%%%%%%%%%%%%%%%%%%%%%%%%%%%%%%%%%%%%%%%%%%%%%%%%%%%%%%%

%%%%%%%%%%%%%%%%%%%%%%%%%%%%%%%%%%%%%%%%%%%%%%%%%%%%%%%%%%%%%%%%%%%%%%%%%
\section{Treść zadania}
Proszę napisać program umożliwiający ukrycie / odczyt w/z pliku BMP innego pliku dowolnego typu.
Program powinien przyjmować liczbę bitów każdej ze składowych RGB, które zostaną przeznaczone na ukrycie informacji.
Program powinien działać dla różnych głębi kolorów. Program uruchamiany z linii
komend z przełącznikami: -i plik do ukrycia, -p plik obrazu, -o plik wyjściowy, -r odczyt ukrytej informacji, -b liczba bitów składowej RGB.

Uwagi:  Jako strukturę danych proszę zastosować tablicę dynamiczną. 
%%%%%%%%%%%%%%%%%%%%%%%%%%%%%%%%%%%%%%%%%%%%%%%%%%%%%%%%%%%%%%%%%%%%%%%%%
\section{Analiza zadania}

Zagadnienie przedstawia problem zapisania dowolnego pliku na ostatnich n bitach składowych tablicy pixeli obrazu w formacie BMP.

\subsection{Struktury danych}
Program wykorzystuje tablice dynamiczne do przechowywania poszczególnych składowych pixeli oraz bajtów plików do zapisania. Wykorzystuje także struktury przechowujące
nagłówki plików BMP oraz specjalny dodatkowy nagłówek steganograficzny, który przechowuje informacje o tym, czy dany plik zawiera ukryte informacje oraz rozmiar ukrytego pliku.

\subsection{Algorytmy}
Program ukrywa pliki dowolnego typu w podanym pliku BMP. Nie wykorzystałem żadnych skomplikowanych algorytmów, jedynie proste operacje mna maskach bitowych, które
umożliwiły mi zapis oraz odczyt danych do oraz z pliku BMP. 

%%%%%%%%%%%%%%%%%%%%%%%%%%%%%%%%%%%%%%%%%%%%%%%%%%%%%%%%%%%%%%%%%%%%%%%%%
\section{Specyfikacja zewnętrzna}
\label{sec:sp:zewnetrzna}
Program jest uruchamiany z linii poleceń. Po podaniu flagi -h wyświetla się pomoc, która podpowiada prawidłową semantykę wywołania programu.



%%%%%%%%%%%%%%%%%%%%%%%%%%%%%%%%%%%%%%%%%%%%%%%%%%%%%%%%%%%%%%%%%%%%%%%%%
\section{Specyfikacja wewnętrzna}\label{sec:sp-wew}
jjhh
 

\subsection{Ogólna struktura programu}
W funkcji głównej wywołana jest funkcja \lstinline|pobierzParametry|.
Funkcja ta sprawdza, czy program został wywołany w prawidłowy sposób. Gdy program nie został wywołany prawidłowo, zostaje wypisany stosowny komunikat i program się kończy.
Następnie wywoływana jest funkcja \lstinline|wczytajDoDrzewa|.
Funkcja ta otwiera plik wejściowy, sczytuje liczby i umieszcza je w drzewie binarnym.
Po sczytaniu wszystkich liczb funkcja zamyka plik.
W razie wystąpienia błędu funkcja zwraca puste drzewo, w przeciwnym wypadku – poprawną wartość korzenia drzewa.
Następnie wywoływana jest funkcja \lstinline|zapiszDrzewoDoPliku|.
Funkcja przechodzi rekurencyjnie drzewo i zapisuje posortowane wartości do pliku wyjściowym. Po zapisaniu liczb funkcja zamyka plik. W razie wystąpienia błędu funkcja zwraca \lstinline!false!, w przeciwnym wypadku -- \lstinline!true!.
Ostatnią funkcją programu jest funkcja zwalniająca pamięć \lstinline|usunDrzewo|.


\subsection{Szczegółowy opis typów i funkcji}

Szczegółowy opis typów i funkcji zawarty jest w załączniku.

 

\section{Testowanie}
Program został przetestowany na różnego rodzaju plikach. Pliki niepoprawne (niezawierające liczb, zawierający liczby w niepoprawnym formacie, niezgodne ze specyfikacją) powodują zgłoszenie błędu. Plik pusty nie powoduje zgłoszenia błędu, ale utworzenie pustego pliku wynikowego (został podany pusty zbiór liczb i pusty zbiór został zwrócony). Maksymalna liczba akceptowana w pliku zależy od kompilatora (typ \lstinline!int! może być realizowany jako zmienna dwu- lub czterobajtowa). Maksymalna wielkość pliku, dla której udało się poprawnie uruchomić program, to \mbox{1.57$\,$GB}. Większe pliki wejściowe powodują błąd alokacji pamięci.

Program został sprawdzony pod kątem wycieków pamięci.


\section{Wnioski}
Program do sortowania liczb jest programem prostym, chociaż wymaga samodzielnego zarządzania pamięcią. Najbardziej wymagające okazało się usunięcie wycieków pamięci. Szczególnie trudne było zapewnienie prawidłowego zwolnienia zaalokowanej pamięci, gdy część danych została wczytana do drzewa, po czym w pliku pojawiały się nieprawidłowe dane. Wtedy program powinien przerwać wczytywanie danych i wyświetlić komunikat, niezaniedbując zwolnienia pamięci.

Dla pewnych danych program wykonywał się poprawnie na niektórych komputerach, podczas gdy na innych maszynach generował niepoprawne wyniki. Było to spowodowane tym, że w zależności od kompilatora typ \lstinline!int! ma 2 albo 4 bajty. Kompilacja typu jako dwubajtowego skutkowała przepełnieniem zakresu zmiennej w czasie wykonania.
 

\begin{filecontents}{bibliografia.bib}
@article{Oetiker2007NieZaKrotkie,
	author       = {Tobias Oetiker and Tomasz Przechlewski and Ryszard Kubiak},
	title       = {Nie za kr\'{o}tkie wprowadzenie do systemu {L}a{T}e{X}2e}},
	YEAR         = {2007},
}

@ARTICLE{id:Abonyi2002modifiedgathgevafuzzy,
  author = {Abonyi, J\'{a}nos and Babu\v{s}ka, Robert and Szeifert, Ferenc},
  title = {{Modified Gath-Geva fuzzy clustering for identification of Takagi-Sugeno
	fuzzy models}},
  journal = {Systems, Man, and Cybernetics, Part B, IEEE Transactions on},
  year = {2002},
  volume = {32},
  pages = {612--621},
  number = {5}
}

@BOOK{id:Cormen2001Wprowadzenie,
  title = {Wprowadzenie do algorytm\'{o}w},
  publisher = {Wydawnictwa Naukowo-Techniczne},
  year = {2001},
  author = {Thomas H. Cormen and Charles E. Leiserson and Ronald L. Rivest},
  address = {Warszawa}
}
\end{filecontents}


\bibliographystyle{plplain}
\bibliography{bibliografia}


 
\cleardoublepage

\rule{0cm}{0cm}

\vfill

\begin{center}
\Huge\bfseries Dodatek\\Szczegółowy opis typów i~funkcji\par
\end{center}

\vfill 

\rule{0cm}{0cm}

\end{document}
% Koniec wieńczy dzieło.
